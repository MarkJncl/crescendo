\chapter{ChessWay Simple} \label{chap:chesswaysimple}
\section{Case Description} The ChessWay is a two-wheeled,
self-balancing scooter which utilises a gyroscope to maintain a stable
upright position. The scooter consists of a platform on which the
rider stands, two parallel wheels and a handlebar.  A model of the
scooter can be seen in Fig~\ref{fig:chessWaySimple}.

\begin{figure}[!ht] \centering
\includegraphics{chessWaySimple/chesswaySimple.JPG}
\caption{Image of the ChessWay personal scooter
\label{fig:chessWaySimple}} \end{figure}

The ChessWaySimple model is a simplified and abstracted model of the
scooter.  It features a sensor which can be used to calculate current
forward velocity and an acceleration sensor which can be used to
determine the current angle. Each wheel has its own motor, although
there is a single controller for both motors, so the scooter travels
in a straight line only (see the ChessWay \DESTECS model in Chapter
\ref{chap:chesswaydestecs} for a slightly more complex model that
permits separate inputs to motors). Users simply lean forwards to
increase forward motion.  This is a trivial and simple version of the
ChessWay model; for more advanced versions of the same vehicle we
recommend studying Chapters \ref{chap:chesswaysl} and
\ref{chap:chesswaydestecs}.

%\section{External Links}

%\begin{itemize} %\item Papers/technical reports where the model
%is used %\end{itemize}

\section{Contract} The contract contains four shared state
variables of type {\textbf\ttfamily{real}}. Two monitored variables
are responsible for reading in signals from the sensors used for
calculating velocity (\texttt{v\_in}) and acceleration
(\texttt{a\_in}). Two controlled variables are responsible for
producing a signal to the motors to alter forward velocity
(\texttt{v\_out}) or to make adjustments to the current angle
(\texttt{a\_out}). Both of the controlled variables produce a signal
to power the motors.

\section{Discrete-event} The DE model includes a
\texttt{Controller} class which manages the main thread of
control. The Controller instantiates two \texttt{IActuatorReal}
abstract classes, to represent the actuator signal for adjusting
vertical position (\texttt{acc\_out}) and the actuator signal for
adjusting forward velocity (\texttt{vel\_out}). It also instantiates
two \texttt{ISensorReal} abstract classes to represent input signals
from the velocity sensor (\texttt{vel\_in}) and the acceleration
sensor (\texttt{acc\_in}). At run-time, the \texttt{Actuator} class
provides a concrete implementation of \texttt{IActuatorReal} and the
\texttt{Sensor} class provides an implementation of
\texttt{ISensorReal}.

The ChessWay scooter implements a closed feedback loop to produce the
self-balancing functionality. The current angle at any one time is
calculated (based on the input from the \texttt{acc\_in} class that
represents the gyroscope's readings), and then based on this, the
scooter's motor is fed a signal to make corrections and ensure the
scooter stays upright.

The \texttt{System} class deploys the \texttt{Controller} onto a
single CPU.

\section{Continuous-time} On the top level of the hierarchy, the
model consists of three main blocks: the controller (the
\texttt{cosim} part of the model), the plant and a block used for
\texttt{IO} which resides in between the \texttt{cosim} and the
plant. The controller handles the passing of variables to/from the DE
model. The plant block contains three elements, to represent the axis,
the person frame (the unit consisting of the platform and handlebar)
and the wheels. And finally the \texttt{IO} block primarily contains
D-A and A-D conversion functionality.

\section{Usage}
The ChessWay Simple model is designed to be a simple introduction to a
more complex version of the same scooter (see Chapter
\ref{chap:chesswaydestecs}).  Running a simulation of the Simple model
demonstrates how the scooter responds to small changes in velocity by
altering signals to the actuators so that it can maintain an upright
position by way of many small corrections.
