\chapter{ChessWay \DESTECS} \label{chap:chesswaydestecs}
\section{Case Description} The ChessWay \DESTECS model is an
extension of the previous two models (see Chapters
\ref{chap:chesswaysimple} and \ref{chap:chesswaysl}). Like the model
presented in Chapter \ref{chap:chesswaysl}, the vehicle described by
the ChessWay\DESTECS model is a self-balancing scooter with two parallel
wheels and a handlebar, an ignition switch and a removable safety key.

The scooter employs sensors to indicate whether the safety key has
been removed from the handlebar unit, or whether the user has switched
off the ignition. There are some differences in how the scooter should
respond when ignition is switched off, or when the safety key is
removed. If the user turns off the scooter, then the model should
first check the current state of the vehicle, and if it is currently
in active motion it should bring it to a safe, stationary position
before deactivating. In contrast, if the safety key is removed from
the handlebar unit, then we assume that the user has fallen from the
scooter and so it should be immediately brought to rest with no need
to check current status.

Unlike the ChessWaySimple model, there are separate controllers for
each of the wheels.  The controllers for the motors are connected
wirelessly, so the model must cope with occasional lost data packets.
As before, there is a sensor which can be used to calculate current
forward velocity of the scooter, and a gyroscope (acceleration) which
can be used to determine the current vertical angle.

\section{Contract} The contract contains six shared state variables of
type {\textbf\ttfamily{real}}. Two monitored variables are responsible for
reading in detected error in velocity
(\texttt{rmsVelocityError}) and the current angle of the
handlebar mast (\texttt{poleAngle}). Four more controlled
variables are responsible for producing signals to the motors to
alter velocity for the left (\texttt{pwmSettingL}) and right
(\texttt{pwmSettingR}) wheels, and for powering the safety key
override on the left (\texttt{safetySwitchL}) and on the right
(\texttt{safetySwitchR}).

The contract also contains 6 display variables, all of type
{\textbf\ttfamily{controlled real}}. The display variables are shared
between the DE and CT models and are available for 3D modelling,
if needed.

Finally, there are two shared design parameters, \texttt{pidGain} and
\texttt{sdpSlip}, the values of which can be set or altered to
simulate different errors.

\section{Discrete-event}  We recommend also reading Chapter \ref{chap:chesswaysl}, which describes the process of implementing this model in more detail.   The DE model includes a number of
classes to designed to represent various sensors, including one each
for \texttt{Gyro}, \texttt{Ignition}, \texttt{SafetyKey},
\texttt{VelocitySensor} and one to detect \texttt{VelocityError} (it's
assumed that the hardware is capable of reporting self-detected
errors). There are also two classes representing actuators:
\texttt{Pid} handles the standard actuators powering the wheels. And
\texttt{SafetySwitch} handles an override of the \texttt{Pid} class,
if the safety key has been withdrawn.

The DE model starts with a \texttt{System} class, which creates
single instances of each of the sensors for monitoring: the
gyroscope; the safety key; the ignition switch; the velocity;
and detected velocity errors. \texttt{System} then creates two
instances of a \texttt{Monitor} class (one for each wheel/motor)
to monitor these sensors and determine the correct current state
for the scooter. \texttt{System} also creates two instances of
the \texttt{RTControl} class, which act as controllers (one for
each motor/wheel). The \texttt{RTControl} instances each have
their own copy of a \texttt{Pid} class (representing the
actuator for activating the motor) and a \texttt{SafetySwitch}
class (for overriding the actuator).

Finally, \texttt{System} also creates a single instance of a
class that represents the wireless data connection. An abstract
class \texttt{Ether} is provided for this purpose along with
several concrete implementations, each of which behaves slightly
differently. Currently the model uses the concrete
implementation \texttt{LossyEther}, which `drops' randomly
selected data packets at runtime.

\section{Continuous-time} The CT model is presented in more
detail than other examples, and several possible 20-sim models
are provided. There are some small variations in detail in each
of the 20-sim models, but for each one the blocks are gathered
into collections of actuators (which receive signals for the
motors driving the wheels and for the safety switch circuit) and
sensors (which produce outputs relating to the current angle,
the velocity and detected error). To maintain a high level of
dependability, we assume that the safety key functionality is
deployed onto a separate CPU to the main plant. Both the main
plant and the safety monitor receive copies of the sensor input;
should the safety monitor detect anything amiss it overrides the
main plant's outputs with its own signal to deactivate the
scooter.

A single block (\texttt{sbsInterface}) handles imports from, and
exports to, the DE model; this block produces outputs for the actuators and
receives inputs from the sensors. In each model a collection of
\texttt{Cycloid} blocks is used to calculate whether the
gyroscope is accurately detecting vertical. A single block
(either \texttt{Chessway} or \texttt{SBS}) represents a detailed
submodel of the plant, and interacts with the actuators, sensors
and error-detecting functionality.

The actuators for each wheel communicate via a wireless
connection in this model, allowing some fault tolerance
behaviour to be tested as the model must cope with packets
occasionally lost in transmission. In the CT model each of the
receivers for the motors is linked to a fault-generating block
that loses occasional packets.

\section{Usage}
This is a complex model, and many scenarios are also possible.  The
\DESTECS co-model provides alternative implementations of the
\texttt{Ether} class, for example, which exhibit different behaviours
and may be tried separately to test fault tolerance.  The difference
between activating the safety key and deactivating the ignition can be
seen by changing the script above to shutdown the ignition instead of
removing the safety key.

\subsection{Scenarios}
A simple scenario is provided for the
ChessWay \DESTECS model, demonstrating some aspects of the
scooter's safety.

\begin{dcl}[caption=DCL script which activates the scooter and then simulates the removal of the safety key and its replacement]
when time = 0.0 do (de real safetyKeyEnabled := 1.0; );

when time = 0.2 do (de real safetyKeyEnabled := 0.0;);

when time = 0.3 do (de real safetyKeyEnabled := 1.0;);

when de real poleAngle < 0.0-0.2
	or de real poleAngle > 0.2
	do ( print "QUIT ... "; quit; );

\end{dcl}

In this scenario, the safety key is withdrawn from the handlebar
unit (\texttt{when time = 0.2}) and then quickly replaced
(\texttt{when time = 0.3}). The scooter responds to the removal
of the safety key by beginning to shut down, but the replacement
of the key results in abandoning the shut down and the scooter
attempting to recover its stable vertical position. The timings
provided in this scenario should demonstrate that the scooter is
capable of recovering and returning to vertical safely. The
scenario can be altered and executed with different timings,
allowing us to determine the maximum length of time that may
elapse between withdrawing the safety key and replacing it and
still have the scooter recover its upright position.

Another possibility is to try out the gyroscope fault detection
functionality.  The script below simulates the detection of an error
in the gyroscope.

\begin{dcl}[caption=DCL script which activates the scooter and then simulates detection of a fault in the gyroscope]
when time = 0.0 do (de real safetyKeyEnabled := 1.0; );

when time = 0.1 do (de int gyroError := 1;);

when time = 0.13 do (de int gyroError := 0;);
\end{dcl}
