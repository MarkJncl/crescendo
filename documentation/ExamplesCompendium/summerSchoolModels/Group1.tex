\section{Group 1}

\subsection{Case Description}
Group 1 opted for a design with two sensors (left and right), placed
symmetrically.  The layout chosen is not completely finished and may
not handle all maps equally well - it may cope with some angles and
corners better than others.  Group 1's model is an example of a model
that prioritises speed when executing turns, as both wheels continue
to turn as fast as is practical whilst turning.  Some other models
made different decisions and either halted one wheel entirely or
reversed one wheel to execute a turn.

\subsection{Contract} The contract contains six variables of type
\keyw{real}. Two \emph{monitored} variables, \texttt{leftLSValye} and
\texttt{rightLSValue}, are used to represent inputs from the
line-detecting sensors on the left and the right respectively.  Two
further \emph{monitored} variables, \texttt{leftEncoderValue} and
\texttt{right\-Encoder\-Value}, are used to represent input from the
sensors used to count revolutions of the wheels.  This information is
necessary in order to calculate distance travelled, a requirement of
the exercise.

Two \emph{controlled} variables of type \keyw{real},
\texttt{leftMotorDC} and \texttt{rightMotorDC}, are used to produce
signals to the motor for the left and right wheels respectively.

The contract also includes the five shared design parameters described
for all groups in Section \ref{chap:summerschoolintro}.

\subsection{Discrete-event}
In the DE model, the \texttt{System} class instantiates two instances
of the \texttt{Sensor} class to represent sensors placed on the left
(\texttt{leftLineSensor}) and the right (\texttt{rightLineSensor}).
In addition \texttt{System} instantiates two instances of
\texttt{Actuator} (\texttt{leftPWM} and \texttt{rightPWM}) to provide
signals to the motors for the left and right wheels.

To provide a closed feedback loop, two more \texttt{Sensor} instances
are created to represent the encoders monitoring actual revolutions
turned by the wheels on the left (\texttt{leftEncoder}) and on the
right (\texttt{rightEncoder}).

Finally, \texttt{System} creates an instance of
\texttt{ModalController} to provide controller functionality, and
deploys it onto a single CPU.  \texttt{ModalController} provides the
primary logic for the robot.  A number of modes are defined; an
initial mode sees the robot start moving forward, hunting for the
initial pattern.  Next is a calibration phase where the robot
continues as it follows the initial pattern, and a line-following
phase where the actual line begins.  A final phase sets the robot at
rest after the end of the line is identified.  The robot reduces the
speed of the left wheel to turn left if the left sensor detects the
line on the ground, and it reduces the speed of the right wheel to
turn right if the right sensor detects the line.

\subsection{Continuous-time}
As with all the other groups, the CT model includes blocks to provide
an implementation of robot physics (\texttt{RobotPhysics}) and data
for the 3D simulation (\texttt{DataFor3D}).

A \texttt{controller} class handles interaction with the DE model.
The CT model includes two blocks to represent the encoders for the
wheels on the left (\texttt{encoderLeft}) and on the right
(\texttt{encoderLeftRight}). Each encoder accepts input (data representing
the number of black or white areas on the wheel which have moved past
the sensor), and incorporates blocks that convert this data into a
count of the number of revolutions the wheel has made.

Input from the \texttt{controller} is passed to \texttt{RobotPhysics}
for some processing and is then made available to the encoders.
Output from \texttt{RobotPhysics} is also passed to a block
\texttt{pos\-Con\-ver\-ter} which is responsible for calculating current
position.  This block interacts with blocks \texttt{floorMapLeft} and
\texttt{FloorMapRight}, which act as lookup tables for interpreting what is
currently visible in the map.



%\subsection{Scenarios}


